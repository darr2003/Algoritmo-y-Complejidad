
El diseño y análisis de algoritmos es una de las áreas más importantes en Ciencias de la Computación. Nos permite comprender cómo funcionan las soluciones a problemas reales y encontrar formas de hacerlas más rápidas y eficientes. Este campo ayuda a mejorar el rendimiento de los programas y que asegura que puedan manejar grandes cantidades de datos o funcionar bien incluso cuando los recursos son limitados. Un ejemplo interesante es el estudio de problemas relacionados con la transformación de cadenas, como calcular el costo más bajo para convertir una cadena en otra usando ciertas operaciones. Este tipo de problemas tiene aplicaciones prácticas en cosas como la edición de texto, lo que lo hace especialmente valioso.
\newline
A lo largo del tiempo, se han creado diferentes formas de abordar estos problemas. Al principio, se usaban métodos sencillos, como los algoritmos de fuerza bruta, que aunque fáciles de implementar, se vuelven poco eficientes cuando las entradas son muy grandes. La programación dinámica ofreció una mejora al almacenar y reutilizar resultados intermedios, lo que hace que los algoritmos sean mucho más rápidos y consuman menos recursos. 
\newline
Este informe tiene como objetivo analizar y comparar dos algoritmos para la transformación de cadenas: el de fuerza bruta y el de programación dinámica. La idea principal es que, aunque la programación dinámica es generalmente más eficiente para cadenas largas, el enfoque de fuerza bruta podría ser más competitivo en casos específicos, como cuando las entradas son pequeñas o las cadenas tienen mucha similitud entre sí. Para poner a prueba esta idea, se implementarán ambos métodos, se medirán sus tiempos de ejecución en diferentes situaciones y se evaluará qué tan precisos son al calcular los costos mínimos de transformación.