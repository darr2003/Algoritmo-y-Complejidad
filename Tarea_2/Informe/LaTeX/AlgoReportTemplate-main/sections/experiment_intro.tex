
\epigraph{``\textit{Non-reproducible single occurrences are of no significance to
science.}''}{---\citeauthor{popper2005logic},\citeyear{popper2005logic} \cite{popper2005logic}}

Para la experimentación y evaluación del desempeño del algoritmo propuesto, se utilizó un equipo con las siguientes especificaciones técnicas:

\begin{itemize}
    \item \textbf{Sistema Operativo:} Windows 10 Home Single Language, 64 bits (versión 10.0, compilación 19045).
    \item \textbf{Procesador:} Intel® Core™ i5-10300H CPU @ 2.50GHz (8 núcleos).
    \item \textbf{Memoria RAM:} 16 GB (16384 MB).
    \item \textbf{Disco Duro:} Unidad NVMe SSD de 512 GB.
    \item \textbf{Tarjeta Gráfica:} NVIDIA GeForce GTX 1650. 
\end{itemize}

 

El programa utiliza las librerias estandar de c++, el uso de la libreria <string>, para manejo de los string de entrada. Especificamente en el programa de fuerza bruta, se utiliza la libreria <limits>, ya que se utiliza para conseguir el valor maximo perimitido por un int. En la solucion por programacion dinamica se utiliza la lirberia <vector> para lacreacion y manejo de la matriz para la resolucion del problema. Ademas de que se uso la libreria <chrono> para el calculo temporal de las funciones a la hora de resolver los distintos casos. Se utilizó un entorno WSL para la ejecucion de los programas.
