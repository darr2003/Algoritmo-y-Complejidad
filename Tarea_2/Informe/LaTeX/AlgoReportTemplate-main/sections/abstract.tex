Este informe presenta dos enfoques para resolver el problema de transformación de cadenas: fuerza bruta y programación dinámica. La implementación en fuerza bruta utiliza recursión para explorar todas las combinaciones posibles de operaciones (sustitución, inserción, eliminación y transposición), pero su complejidad exponencial la hace impráctica para cadenas largas. En contraste, la programación dinámica optimiza el proceso utilizando una matriz para almacenar costos intermedios, logrando una mayor eficiencia.

La evaluación experimental se realizó en un equipo con especificaciones específicas, utilizando pruebas de tiempo y validación de costos. Los resultados muestran que la programación dinámica es considerablemente más rápida y eficiente en términos de memoria, especialmente para cadenas más largas.