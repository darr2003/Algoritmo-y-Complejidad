

En conclusión, los resultados muestran claramente las diferencias en el rendimiento entre los enfoques de fuerza bruta y programación dinámica para la transformación de cadenas. La solución de fuerza bruta, aunque efectiva en pequeños conjuntos de datos, tiene una complejidad exponencial que la hace ineficiente para cadenas más largas. Por otro lado, la programación dinámica ofrece una solución mucho más eficiente, con una complejidad temporal de O(m × n), lo que la hace ideal para aplicaciones prácticas. Este análisis resalta la importancia de elegir el enfoque adecuado según el tamaño y las características de los datos, lo que mejora considerablemente el rendimiento en situaciones reales. Las pruebas experimentales confirmaron que la programación dinámica no solo reduce el tiempo de ejecución, sino que también optimiza el uso de memoria, lo que la hace más adecuada para trabajar con grandes volúmenes de datos. Aunque ambos métodos tienen su utilidad, la programación dinámica destaca como la opción preferida cuando se busca eficiencia y escalabilidad.