\subsection*{Pruebas realizadas}

Para la experimentación actual, implementamos dos tipos principales de pruebas:

\begin{enumerate}
    \item \textbf{Prueba temporal:}  
    En esta prueba, nos enfocamos en medir el tiempo que cada método (fuerza bruta y programación dinámica) tarda en resolver una serie de ejercicios generados mediante programas en Python, independientemente del resultado de la función. Los casos evaluados incluyen:  
    \begin{itemize}
        \item \textbf{Cadenas de distintos tamaños}
        \item \textbf{Cadenas completamente iguales}
        \item \textbf{Una cadena vacía y otra no vacía}  
        \item \textbf{Cadenas de igual tamaño pero con letras completamente distintas}  
        \item \textbf{Entradas que solo pueden resolverse mediante transposición}  
    \end{itemize}
    Este enfoque permitió comparar directamente los tiempos de ejecución entre ambos métodos en diferentes escenarios.

    \item \textbf{Prueba de validación de costos:}  
    En esta prueba utilizamos un dataset limitado con entradas seleccionadas y cuales resultados esperados fueron calculados manualmente. Posteriormente, comparamos estos valores con los resultados obtenidos por el programa, asegurándonos de que los costos fueran correctos en todos los casos.
\end{enumerate}

Estas pruebas nos permitieron analizar la eficiencia en términos de rendimiento.

\subsubsection*{Ejemplos de entradas para pruebas}

A continuación, se presentan ejemplos de entradas utilizadas para evaluar el rendimiento y la precisión del algoritmo, clasificados según las características de las cadenas:

\begin{enumerate}
    \item \textbf{Cadenas de distintos tamaños:}
    \begin{itemize}
        \item Entrada:
        \begin{verbatim}
        "st"
        "abcdef"
        \end{verbatim}
        \item Explicación: Una cadena es significativamente más grande que la otra, lo que genera múltiples inserciones.
    \end{itemize}

    \item \textbf{Cadenas completamente iguales:}
    \begin{itemize}
        \item Entrada:
        \begin{verbatim}
        "algoritmo"
        "algoritmo"
        \end{verbatim}
        \item Explicación: Ambas cadenas son idénticas, lo que debería resultar en un costo de transformación igual a cero.
    \end{itemize}

    \item \textbf{Una cadena vacía y otra no vacía:}
    \begin{itemize}
        \item Entrada:
        \begin{verbatim}
        ""
        "transformar"
        \end{verbatim}
        \item Explicación: Una de las cadenas está vacía, lo que implica que todas las operaciones serán inserciones.
    \end{itemize}

    \item \textbf{Cadenas de igual tamaño pero con letras completamente distintas:}
    \begin{itemize}
        \item Entrada:
        \begin{verbatim}
        "abcde"
        "vwxyz"
        \end{verbatim}
        \item Explicación: Ambas cadenas tienen el mismo tamaño, pero no tienen caracteres en común, por lo que el resultado implicará únicamente sustituciones.
    \end{itemize}

    \item \textbf{Entradas que solo pueden resolverse mediante transposición:}
    \begin{itemize}
        \item Entrada:
        \begin{verbatim}
        "ab"
        "ba"
        \end{verbatim}
        \item Explicación: Las cadenas son iguales en contenido pero tienen los caracteres intercambiados, lo que hace necesario el uso de la operación de transposición.
    \end{itemize}
\end{enumerate}
